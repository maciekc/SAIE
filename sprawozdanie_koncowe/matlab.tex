\chapter{Implementacja}
\label{implementacja}
Do dobrania nastaw regulatorów wykorzystano funkcje optymalizującą \textit{fmincon}, dostępną w środowisku Matlab/Simulink. Funkcja ta bazując na modelu obiektu (rys. \ref{model_systemu}), minimalizowała zadany wska\'znik jakości, który w tym przypadku był podany wzorem \ref{wsk_jak}. \\
Na listingu \ref{cel_matlab} zaprezentowano kod funkcji wywoływanej przez procedurę optymalizującą. Aby umożliwić poprawne działanie procedury \textit{fmincon} należało do funkcji wywołującej model Simulika dodać linijkę
\begin{center}
	\textit{opt = simset('SrcWorkspace','Current').}
\end{center}
 Dzięki wywołaniu tej instrukcji optymalizacja korzystała ze zmiennych lokalnego workspace-a, co umożliwiło uruchamianie modelu dla kolejnych zbiorów parametrów.
\begin{lstlisting}[frame=single, caption=Funkcja wywoływana przez procedurę optymalizującą., label = cel_matlab]
function wsk = cel( par )

	global Kr Kw1 Tw2 Kw2 To Ko z1 z2 r tau limit1 limit2 limit3
	P1 = par(1);
	D1 = par(2);
	P2 = par(3);
	D2 = par(4);
	P3 = par(5);
	I3 = par(6);
	Kr = par(7);
	
	opt = simset('SrcWorkspace','Current');
	sim('model',50,opt)
	
	wsk = J;
end
\end{lstlisting}
\newpage

Kod procedury odpowiedzialnej za przeprowadzenie całego procesu optymalizacji znajduje się na listingu \ref{opt_kod}
 \lstset { %
	language=Matlab
}
\begin{lstlisting}[frame=single, caption=Procedura optymalizacji, label=opt_kod]
clear all
close all


global Kr Kw1 Tw2 Kw2 To Ko z1 z2 r tau limit1 limit2 limit3

% Wartosci poczatkowe parametrow regulatorow
P1 = 0.01;
D1 = 0.5;

P2 = 1;
D2 = 0.5;

P3 = 1;
I3 = 0.01;

P4 = 1;

Kr = 1.3;

% Parametry obiektu
Kw1 = 10;

Tw2 = .1;
Kw2 = 5;
To = 1;
Ko = 10;

tau = 1;
limit1 = 20;
limit3 = 40;
limit2 = 40;

% Wartosci zadane
zad = [5 10 20 50 70]; 


X0 = [P1 D1 P2 D2 P3 I3 Kr];
A = -eye(7);
B = zeros(7,1);
LB = [0 0 0 0 0 0 0];
UB = [1e5 1e5 1e5 1e5 1e5 1e5 1e5];

options = optimoptions('fmincon','Display','iter','MaxIterations',100);

zad = [5 10 20 50 70];

Parametry = [];
for i = 1:5
	f = inf;
	par = [];
	r = zad(i);
	[par f] = fmincon(@cel, X0, A, B,[],[],LB,UB,[],options)
end

Parametry(i,:) = par;
end
\end{lstlisting}
