\chapter{Wnioski}

% Zamodelowano nieznany układ opisany modelem Strejca pierwszego, drugiego oraz trzeciego rzędu, a następnie przeprowadzono optymalizację nastaw regulatorów w obecności dwóch zakłóceń - jednego mierzalnego, drugiego niemierzalnego. Model odwzorowywał układ grzewczy, na wejście którego podawana była pewna temperatura mająca się utrzymać na jego wyjściu. Zastosowanie regulatora typu $feedforward$ ($PD_1$) pozwoliło stworzyć szybki układ nadążający o dużej dokładności. Dla tego regulatora im mniejszy rząd modelu tym mniejsze przeregulowania. Wartości wskaźnika jakości rosną wraz ze wzrostem wartości zadanej oraz rzędu modelu, jednak ich stosunek w obrębie jednego rzędu jest podobny, co wskazuje na znalezienie optymalnych nastaw dla każdego zestawu parametrów. Część całkująca regulatora proporcjonalno-całkującgo $PI$ działa jedynie dla dużych wartości zadanych, co jest związane z dużą stałą czasową. Zastosowanie regulatora proporcjonalnego $P$ w tym miejscu w większości przypadków byłoby wystarczające. Regulator ten odpowiada na znane zakłócenia, dzięki czemu można go łatwo nastroić oraz uprościć jego budowę, co zmniejszyłoby koszt całej instalacji. Jak łatwo zauważyć na podstawie wykresów, cel zadania został osiągnięty z satysfakcjonującym wynikiem. Układ szybko reaguje zarówno na znane, jak i nieznane zakłócenia, a tworzące się przeregulowania są niewielkie i nie wpływają znacząco na działanie całego układu.
Biorąc po uwagę zamieszczone przebiegi odpowiedzi obiektu dla znalezionych nastaw regulatora rys. \ref{wykres_1} - \ref{wykres_4} można wysnuć wniosek, że część różniczkująca w regulatorze $PD_1$ ma pomijalnie mały wpływ na dynamiką układu. Związane jest to z tym, że generuje ona sterowanie tylko w przypadku zmiany wartości zadanej. Dodatkowo, sterowanie to ma postać pojedynczej szpilki, która nie wpływa na działanie obiektu biorąc pod uwagę jego dużą inercję. Dużo większy pływ na obiekt ma za to część proporcjonalna wspomnianego regulatora. Generuje ona bowiem natychmiastową poprawkę sterowania w zależności od aktualnej wartości zadanej. Działanie to poprawia dynamikę układu regulującego, który bazując tylko na regulatorze $PI$, potrzebowałby więcej czasu na korekcję sygnału sterującego (czas potrzebny na redukcję całki). \\   
Analizując wartości wzmocnienia części całkującej w tabelach \ref{par_reg_zes1} - \ref{par_reg_zes3} można stwierdzić, że regulator $PI$ mógłby zostać z powodzeniem zastąpiony regulatorem proporcjonalnym dla wartości zadanej przyjmującej wartości 5, 10, 20.  Zwiększenie wartości członu całkującego w tych przypadkach skutkowało pojawieniem się znacznego przeregulowania i w konsekwencji oscylacji w odpowiedzi obiektu. Wpływ na takie zachowanie obiektu miało także wprowadzenie saturacji na sygnały wyjściowe z regulatorów i urządzenia wykonawczego. W procesie optymalizacji próbowano ograniczyć oscylacje poprzez wprowadzenie filtru \textit{wind-up} w postaci korekcji wstecznej członu całkującego. \\ 
Z zamieszczonych przebiegów odpowiedzi obiektu rys. \ref{wykres_1} - \ref{wykres_4} wynika, że zaproponowana struktura eliminacji wpływu zakłóceń na działanie całego systemu przyniosła zamierzony efekt. Dodatkowa pętla sprzężenie wokół urządzenia wykonawczego w zadawalający sposób kompensowała zakłócenie $z_1$. Dodatkowy regulator $PD_2$ operujący na wartości zakłócenia $z_2$ natychmiast niwelował jego wpływ na wartość sygnału wyjściowego, co można zaobserwować na załączonych wykresach - po jednej sekundzie (opóźnienie czasowe obiektu) wyjście obiektu było sprowadzane do zadanego poziomu. \\
Podsumowując, cel niniejszego projektu, którym było dobranie nastaw regulatorów w zależności do postaci obiektu i wartości zadanej, zostało zrealizowane.  Jednak biorąc po uwagę wnioski zawarte w powyższych akapitach można przypuszczać, że zastąpienie regulatorów $PI$ i $PD_1$ regulatorami proporcjonalnymi nie powinno znacząco pogorszyć działania całego układu. Operacja ta uprościłaby za to cały proces strojenia.  
