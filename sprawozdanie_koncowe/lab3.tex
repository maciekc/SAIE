\chapter{Wnioski}

 Zamodelowano nieznany układ opisany modelem Strejca pierwszego, drugiego oraz trzeciego rzędu, a następnie przeprowadzono optymalizację nastaw regulatorów w obecności dwóch zakłóceń - jednego mierzalnego, drugiego niemierzalnego. Model odwzorowywał układ grzewczy, na wejście którego podawana była pewna temperatura mająca się utrzymać na jego wyjściu. Zastosowanie regulatora typu $feedforward$ ($PD_1$) pozwoliło stworzyć szybki układ nadążający o dużej dokładności. Dla tego regulatora im mniejszy rząd modelu tym mniejsze przeregulowania. Wartości wskaźnika jakości rosną wraz ze wzrostem wartości zadanej oraz rzędu modelu, jednak ich stosunek w obrębie jednego rzędu jest podobny, co wskazuje na znalezienie optymalnych nastaw dla każdego zestawu parametrów. Część całkująca regulatora proporcjonalno-całkującgo $PI$ działa jedynie dla dużych wartości zadanych, co jest związane z dużą stałą czasową. Zastosowanie regulatora proporcjonalnego $P$ w tym miejscu w większości przypadków byłoby wystarczające. Regulator ten odpowiada na znane zakłócenia, dzięki czemu można go łatwo nastroić oraz uprościć jego budowę, co zmniejszyłoby koszt całej instalacji. Jak łatwo zauważyć na podstawie wykresów, cel zadania został osiągnięty z satysfakcjonującym wynikiem. Układ szybko reaguje zarówno na znane, jak i nieznane zakłócenia, a tworzące się przeregulowania są niewielkie i nie wpływają znacząco na działanie całego układu.